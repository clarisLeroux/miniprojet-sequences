
\documentclass[a4paper, 12pt]{article}

\usepackage[T1]{fontenc}
\usepackage[utf8]{inputenc}
\usepackage[french]{babel}
\usepackage{graphicx}	
\usepackage{color}
\usepackage[left=2cm, right=2cm, top=2cm, bottom=2cm]{geometry} 
\usepackage{amsmath}		
\usepackage{amsfonts}		
\usepackage{amssymb}	
\usepackage{listings}
\usepackage{verbatim}
\usepackage{hyperref}


\usepackage[dvipsnames]{xcolor}
\definecolor{GRIS}{cmyk}{0.7,0.6,0.5,0.3}
\definecolor{BLEU}{cmyk}{1,0.9,0.1,0}

%%configuration de listings
\lstset{
language=[77]Fortran,
basicstyle=\ttfamily\small, 
identifierstyle=\color{black}, 
keywordstyle=\color{blue}, 
stringstyle=\color{black}, 
commentstyle=\it\color{ForestGreen}, 
columns=flexible, 
tabsize=2, 
extendedchars=true, 
showspaces=false, 
showstringspaces=false, 
numbers=left, 
numberstyle=\tiny, 
breaklines=true, 
breakautoindent=true, 
captionpos=b
}

\begin{document}
 

%
% vos diagrammes
%

%
%nos diagrammes
%

\section{initialiser jeu}
 \subsection{Scénario}
 
 \begin{itemize}
  \item \textbf{Contexte} \\
  Instanciation d'un jeu par l'IHM lancée par l'utilisateur
  \item \textbf{Réduction du cas}\\
  Initialisation du jeu à deux joueurs
  \item \textbf{Déclenchement}\\
  Automatique lors de l'instanciation d'un jeu
  \item \textbf{Résultat}\\
  Une instance de jeu dont toute la structure (plateau, cases et pions) a été initialisée et où sont inscrits deux joueurs de nom A et B correctement initialisés  
  \item \textbf{Déroulement}\\
  Créer un plateau (qui sera initialisé) et l'associer à la partie\\
  Initialiser 2 joueurs de nom A et B et les associer à la partie\\
 \end{itemize}

 \subsection{Pseudo code}
 \begin{lstlisting}    
  creer(){
  Partie partie = new Partie();
  nbJoueurs=2;
  creerPlateau(partie);
  initialiserJoueurs(nbJoueurs, partie);
  }
 \end{lstlisting}

 \section{creerPlateau}
  \begin{itemize}
  \item \textbf{Contexte} \\
  \item \textbf{Réduction du cas}\\
  \item \textbf{Déclenchement}\\
  \item \textbf{Résultat}\\
  \item \textbf{Déroulement}\\
 \end{itemize}

 \subsection{Pseudo code}
 \begin{lstlisting}    
 creerPlateau(Partie partie){
 
 }
 \end{lstlisting}
\end{document}

